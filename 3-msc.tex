%------------------------------------------------------------------------------%

\subsubsection{Contexto}

\begin{frame}{Origem}
  O Minimum Scan Cover (MSC) foi introduzido em 2021~\cite{Fe21} como um problema de comunicação ponto a ponto entre satélites:
  \bigbreak
  \begin{minipage}{\linewidth}
    \centering
    \only<1>{\includegraphics[height=5cm]{MSC/instance/Temp-0.png}}%
    \only<2>{\includegraphics[height=5cm]{MSC/instance/Temp-1.png}}%
    \only<3>{\includegraphics[height=5cm]{MSC/instance/Temp-2.png}}%
    \only<4>{\includegraphics[height=5cm]{MSC/instance/Temp-3.png}}%
  \end{minipage}
\end{frame}

\begin{frame}{Motivação}
  \begin{itemize}[<+->]

    \item É possível usar transmissão direcional e recepção omnidirecional;

    \item Mas exige duas antenas, aumentando custo e complexidade.

  \end{itemize}
\end{frame}

\begin{frame}{Definição - Instância}
  \begin{itemize}[<+->]

    \item Conjunto $P = \{p_1, \dots, p_n\}\subseteq \R^d$ de posições distintas;

    \item Grafo $G=(P, E)$ onde $E\subseteq P\times P$.

  \end{itemize}
\end{frame}

\begin{frame}{Definição - Solução}
  \textbf{Escalonamento} ${\cals\colon E\to \R^+}$ onde queremos minimizar $\max\limits_{e \in E} \cals(e)$:
  \begin{minipage}{\linewidth}
    \centering
    \vspace*{1cm}
    \only<1>{\includegraphics[height=4cm]{MSC/solution/Sol-0.png}}%
    \only<2>{\includegraphics[height=4cm]{MSC/solution/Sol-1.png}}%
    \only<3>{\includegraphics[height=4cm]{MSC/solution/Sol-2.png}}%
    \only<4>{\includegraphics[height=4cm]{MSC/solution/Sol-3.png}\bigbreak}
    \only<4>{$\cals(a)=\cals(b)=0$, $\cals(c)=\cals(d)=\frac{\pi}{4}$ e $\cals(e)=\frac{\pi}{2}$.}
  \end{minipage}
\end{frame}

%------------------------------------------------------------------------------%

\subsubsection{Resultados para 1D}

\begin{frame}{Resultado Anterior}
  \begin{thm}[Fekete et al.~\cite{Fe21}]
    Mesmo em 1D, para todo $\gamma \geq 1$, uma $\gamma$-aproximação para o MSC implica que $P = NP$.
  \end{thm}
\end{frame}

\begin{frame}{Nossos Resultados}
  \setbeamercolor{block body}{bg=green!35!white}
  \begin{thm}
    Existe um algoritmo \FPT para o MSC em 1D, parametrizado pela largura de árvore $k$, que roda em tempo $k^{O(k)} \cdot n$.
  \end{thm}

  \pause
  \begin{cor}
    Existe uma $3$-aproximação para o MSC em 1D em grafos planares, que roda em tempo $O(n^2)$.
  \end{cor}
\end{frame}

%------------------------------------------------------------------------------%

\subsubsection{Resultados para 2D}

\begin{frame}{Resultados Anteriores}
  \begin{thm}[Fekete et al.~\cite{Fe21}]
    Mesmo em 2D, para todo $\gamma < \nicefrac{3}{2}$, uma $\gamma$-aproximação para o MSC em grafos bipartidos implica que $P = NP$.
  \end{thm}

  \pause
  \begin{thm}[Fekete et al.~\cite{Fe21}]
    Existe uma $\frac{9}{2}$-aproximação para o MSC em 2D em grafos bipartidos.
  \end{thm}
\end{frame}

\begin{frame}{Preliminares}
  \centering
  Uma instância usando $\ell=8$ direções:

  \bigskip
  \begin{minipage}{\linewidth}
    \centering
    \includegraphics[height=5cm]{MSC/directions.png}
  \end{minipage}
\end{frame}

\begin{frame}{Nossos Resultados}
  \setbeamercolor{block body}{bg=green!35!white}
  \begin{thm}
    Existe um algoritmo \FPT para MSC em 2D, parametrizado pela largura de árvore $k$ e $\ell$, que roda em tempo $k^{O(k \ell \log(\ell))} \cdot n$.
  \end{thm}

  \pause
  \begin{cor}
    Existe uma $3$-aproximação para MSC em 2D em grafos planares parametrizada por $\ell$, que roda em tempo $\ell^{O(\ell)} \cdot n + O(n^2)$.
  \end{cor}
\end{frame}

\begin{frame}{Preliminares}
  \centering
  Uma instância de ângulo mínimo $\lambda$:

  \bigskip
  \begin{minipage}{\linewidth}
    \centering
    \includegraphics[height=2.5cm]{MSC/lambda.png}
  \end{minipage}
\end{frame}

\begin{frame}{Nosso Resultado}
  \setbeamercolor{block body}{bg=green!35!white}
  \begin{thm}
    Existe uma $2$-aproximação para o MSC em 2D parametrizada pela largura de árvore $k$ e $\lceil \nicefrac{1}{\lambda} \rceil$, que roda em tempo $\lambda^{-O(k^2 + \frac{k \log k}{\lambda})} \cdot (\log k)^{O(k^2)} \cdot n$.
  \end{thm}
\end{frame}

\begin{frame}{Rascunho da Prova}
    \begin{itemize}[<+->]
        \item Seja $I = (P, G)$ uma instância com ângulo mínimo $\lambda$;

        \item Recebemos uma decomposição em árvore de $G$ com largura $k$;

        \item \emph{WLOG}, considere que $\lambda < \pi$.
    \end{itemize}
\end{frame}

\begin{frame}{}
    \begin{itemize}[<+->]
        \item Seja $M := \floor{\nicefrac{\pi}{\lambda}} \geq 1$;

        \item O disco ao redor de cada $p_i\in P$ é dividido em $4M$ \textbf{setores}:
    \end{itemize}
    
    \bigskip
    \pause
    \begin{minipage}{\linewidth}
        \centering
        \includegraphics[height=4cm]{MSC/angle.png}
    \end{minipage}

    \bigskip
    \centering
    $\theta := \frac{\pi}{2M} < \lambda$ e os setores são $c_j$ para $j = 0, \dots, 4M-1$.
\end{frame}

\begin{frame}{}
    Relaxação ${\hat I:=I}$ do problema dependendo de $\theta$:
    
    \bigskip
    \pause
    \begin{minipage}{\linewidth}
        \centering
        \includegraphics[height=5cm]{MSC/discrete.png}
    \end{minipage}
\end{frame}

\begin{frame}{Nosso Resultado}
  \setbeamercolor{block body}{bg=green!35!white}
  \begin{cor}
    Existe uma $5$-aproximação para o MSC em 2D em grafos planares parametrizada por $\lceil \nicefrac{1}{\lambda} \rceil$, que roda em tempo $\lambda^{-O(\nicefrac{1}{\lambda})} \cdot n + O(n^2)$.
  \end{cor}
\end{frame}
